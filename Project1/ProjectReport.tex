\documentclass[12pt,a4paper]{report}

% ------------------------------------------------------------
% PACKAGES
% ------------------------------------------------------------
\usepackage[utf8]{inputenc}        % Encoding
\usepackage[T1]{fontenc}           % Font encoding
\usepackage{lmodern}               % Latin Modern font
\usepackage[margin=1in]{geometry}  % Page margins
\usepackage{setspace}              % Line spacing
\usepackage{amsmath, amssymb, amsthm, mathtools}  % Math packages
\usepackage{graphicx}              % For figures
\usepackage{caption}               % Better captions
\usepackage{subcaption}            % Subfigures
\usepackage{booktabs}              % Nicer tables
\usepackage{hyperref}              % Clickable links
\usepackage{enumitem}              % Custom lists
\usepackage{physics}               % For quantum notation, etc.
\usepackage{biblatex}              % Bibliography management
\addbibresource{references.bib}    % Reference file

% ------------------------------------------------------------
% CUSTOM SETTINGS
% ------------------------------------------------------------
\setstretch{1.2}
\hypersetup{
    colorlinks=true,
    linkcolor=blue,
    citecolor=teal,
    urlcolor=magenta
}

% ------------------------------------------------------------
% TITLE INFORMATION
% ------------------------------------------------------------
\title{%
    \textbf{Sample Report Template}\\[1ex]
    \large A Demonstration of a Professional LaTeX Report
}
\author{Your Name \\ Graduate Student, Department of Mathematics}
\date{\today}

% ------------------------------------------------------------
% DOCUMENT BODY
% ------------------------------------------------------------
\begin{document}

\maketitle
\tableofcontents
\newpage

% ------------------------------------------------------------
\chapter{Introduction.}

This report provides an example structure for writing technical or research reports in \LaTeX. It demonstrates common environments such as equations, figures, and references.

\section{Motivation}
Mathematical and computational reports often benefit from clear structure and consistent formatting. This template aims to provide both.

\section{Background}
You can cite references using \texttt{biblatex}, such as Einstein's work on relativity \cite{einstein1905}.

% ------------------------------------------------------------
\chapter{Mathematical Framework}

\section{Equations}
Equations can be displayed inline, e.g. \( E = mc^2 \), or in display mode:
\begin{equation}
    \nabla^2 \psi + k^2 \psi = 0
    \label{eq:helmholtz}
\end{equation}
You can reference Equation~\ref{eq:helmholtz} later.

\section{Definitions and Theorems}

\begin{definition}
A \emph{topological space} is a set \( X \) together with a collection \( \tau \) of subsets of \( X \) satisfying the topology axioms.
\end{definition}

\begin{theorem}
Let \( f: X \to Y \) be continuous. If \( X \) is compact, then \( f(X) \) is compact.
\end{theorem}

\begin{proof}
Let \( \{V_i\} \) be an open cover of \( f(X) \). Then \( \{f^{-1}(V_i)\} \) is an open cover of \( X \), which has a finite subcover by compactness.
\end{proof}

% ------------------------------------------------------------
\chapter{Figures and Tables}

\section{Figures}
You can include figures as shown in Figure~\ref{fig:example}.

\begin{figure}[h!]
    \centering
    \includegraphics[width=0.6\textwidth]{example-image-a}
    \caption{Example figure with a caption.}
    \label{fig:example}
\end{figure}

\section{Tables}
\begin{table}[h!]
    \centering
    \caption{Sample data table}
    \begin{tabular}{lcc}
        \toprule
        Variable & Mean & Std. Dev. \\
        \midrule
        $x_1$ & 1.23 & 0.12 \\
        $x_2$ & 4.56 & 0.45 \\
        $x_3$ & 7.89 & 0.78 \\
        \bottomrule
    \end{tabular}
\end{table}

% ------------------------------------------------------------
\chapter{Discussion}

Discuss your results, implications, and future work here. For example, in quantum computing contexts, one might describe how tensor product structures extend computational models.

% ------------------------------------------------------------
\chapter{Conclusion}

Summarize the key findings or methods. You might include:
\begin{itemize}[leftmargin=2em]
    \item Theoretical insights or results.
    \item Implementation details.
    \item Open questions or future directions.
\end{itemize}

% ------------------------------------------------------------
\printbibliography

% ------------------------------------------------------------
\appendix
\chapter{Appendix A: Additional Material}

Appendices can include code listings, proofs, or detailed data.

\end{document}
